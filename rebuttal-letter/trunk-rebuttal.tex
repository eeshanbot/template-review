\documentclass{article}
	\def\papertitle{Making Dams All Day Every Day}
	\def\authors{E. Beaver, H. Bee, S.N. Mole, S. Shrimp}
	\def\journal{Muddy Charles}
	\def\doi{3.1415}
% Define title defaults if not defined by user
\providecommand{\lettertitle}{Author Response to Reviews of}
\providecommand{\papertitle}{Title}
\providecommand{\authors}{Authors}
\providecommand{\journal}{Journal}

\usepackage[includeheadfoot,top=20mm, bottom=20mm, footskip=2.5cm]{geometry}

% TYPOGRAPHY
\usepackage{cmbright}
\usepackage{amssymb,amsmath}
\usepackage{microtype}
\usepackage[utf8]{inputenc}

% MISC
\usepackage{graphicx}
\usepackage[hidelinks]{hyperref} %textopdfstring from pandoc
\usepackage{soul} % Highlight using \hl{}

% TABLE

\usepackage{adjustbox} % center large tables across textwidth by surrounding tabular with \begin{adjustbox}{center}
\renewcommand{\arraystretch}{1.5} % enlarge spacing between rows
\usepackage{caption} 
\captionsetup[table]{skip=10pt} % enlarge spacing between caption and table

%% SECTION STYLES

\usepackage{titlesec}

% section
\titleformat{\section}{\pagebreak \normalfont\LARGE}{\makebox[0pt][r]{\bf \thesection.\hspace{4mm}}}{0em}{\bfseries}
%\titleformat*{\section}{\pagebreak \LARGE\bfseries}

% subsection

\titleformat{\subsection}{\normalfont}{\makebox[0pt][r]{\bf \thesubsection.\hspace{4mm}}}{0em}{\bfseries}
\titlespacing{\subsection}{0em}{1em}{-0.3em} % left before after
\titleformat*{\subsection}{\large\bfseries}
\titlespacing*{\subsection}
{0em}{1em}{1em}

% subsubsection
\titlespacing*{\subsubsection}
{0em}{0.5em}{0.5em}
\renewcommand\thesubsubsection{\arabic{section}.\arabic{subsubsection}}

%% PARAGRAPH STYLES

\setlength{\parskip}{0.6\baselineskip}%
\setlength{\parindent}{0pt}%

%% QUOTATION STYLES

\usepackage{framed}
\let\oldquote=\quote
\let\endoldquote=\endquote
\renewenvironment{quote}{\begin{fquote}\advance\leftmargini -2.4em\begin{oldquote}}{\end{oldquote}\end{fquote}}

\usepackage{xcolor}
\newenvironment{fquote}
  {\def\FrameCommand{
	\fboxsep=0.6em % box to text padding
	\fcolorbox{black}{white}}%
	% the "2" can be changed to make the box smaller
    \MakeFramed {\advance\hsize-2\width \FrameRestore}
    \begin{minipage}{\linewidth}
  }
  {\end{minipage}\endMakeFramed}

%% TABLE STYLES

\let\oldtabular=\tabular
\let\endoldtabular=\endtabular
\renewenvironment{tabular}[1]{\begin{adjustbox}{center}\begin{oldtabular}{#1}}{\end{oldtabular}\end{adjustbox}}

%% MACROS FOR FORMATTING NICELY

% common
\usepackage{mdframed}
\newmdenv[
  topline=false,
  bottomline=false,
  rightline=false
]{sideline}

\newcommand{\ebline}[1]{
	\begin{sideline}
	\begin{em}
	#1
	\end{em}
	\end{sideline}
}

% rebuttal-letter
\newcommand{\reviewer}[2]{
	\subsubsection{#1}
	\ebline{#2}
}

% reviewer-letter
\newcommand{\bigcomment}[1]{
\subsection{}
\vspace{-3em}
#1
}

%%DIF PREAMBLE EXTENSION ADDED BY LATEXDIFF
%% DIF UNDERLINE PREAMBLE %DIF PREAMBLE
\RequirePackage[normalem]{ulem} %DIF PREAMBLE
\RequirePackage{color}\definecolor{RED}{rgb}{1,0,0}\definecolor{BLUE}{rgb}{0,0,1} %DIF PREAMBLE
\providecommand{\DIFadd}[1]{{\protect\color{blue}\uwave{#1}}} %DIF PREAMBLE
\providecommand{\DIFdel}[1]{{\protect\color{red}\sout{#1}}}                     %DIF PREAMBLE
%DIF SAFE PREAMBLE %DIF PREAMBLE
\providecommand{\DIFaddbegin}{} %DIF PREAMBLE
\providecommand{\DIFaddend}{} %DIF PREAMBLE
\providecommand{\DIFdelbegin}{} %DIF PREAMBLE
\providecommand{\DIFdelend}{} %DIF PREAMBLE
%DIF FLOATSAFE PREAMBLE %DIF PREAMBLE
\providecommand{\DIFaddFL}[1]{\DIFadd{#1}} %DIF PREAMBLE
\providecommand{\DIFdelFL}[1]{\DIFdel{#1}} %DIF PREAMBLE
\providecommand{\DIFaddbeginFL}{} %DIF PREAMBLE
\providecommand{\DIFaddendFL}{} %DIF PREAMBLE
\providecommand{\DIFdelbeginFL}{} %DIF PREAMBLE
\providecommand{\DIFdelendFL}{} %DIF PREAMBLE
%DIF END PREAMBLE EXTENSION ADDED BY LATEXDIFF

%% RANDO TEXT FOR EXAMPLE
\usepackage{lipsum}

%% CHECKS IF THINGS ARE EMPTY
\usepackage{etoolbox}

%% NICE REFERENCING
% must be loaded after hyperref
% puts in section, fig, etc in front of comment
\usepackage[capitalise]{cleveref}

%% CROSS REFERENCE TO ORIGINAL DOCUMENT

% if you just use cite
% \usepackage{xr}
% \externaldocument[v1-]{pathToDoc}
% in doc, do \ref{v1-sec:intro}

% if you use hyperref
% \usepackage{xr-hyper}
% \externaldocument[v1-]{pathToDoc}

\begin{document}

% Make title
{\Large\bf \lettertitle}\\[1em]
{\huge \papertitle}\\[1em]
{\authors}\\
{\emph{\journal}}\ifdefempty{\doi}{}{, \texttt{doi:\doi}}\\
\hrule

% Legend
% \hfill {\vline ~\textit{Reviewer Comment}, Author Response, \(\quad\square\) Manuscript text

\begin{flushright}
\vspace{-1em}
\vline ~ \emph{Reviewer Comment} \\
Author Response \\
$\square$ Manuscript Text
\end{flushright}

% -------- BEGIN REBUTTAL, GOOD LUCK ------- %%

Dear Editors,

We thank the reviewers and editor for their thoughtful feedback and comments in improving this work. The rebuttal letter is organized with sections delineated by the review, shown in the table of contents.

Sincerely,

Dr. Eager Beaver \\
\emph{on behalf of all authors}

% table of contents
\renewcommand{\contentsname}{}
\begingroup
\let\pagebreak\relax
\setcounter{tocdepth}{1}
\tableofcontents
\endgroup

\section{Reviewer One}

\subsection*{Summary}

\ebline{\lipsum[1]}

\lipsum[2]

\subsection*{Comments}

\lreviewer{Line 12}{v1:1.1}
{You say ``Duis eget orci sit amet orci dignissim rutrum''. Please check the grammar.}
This is correct. We have not changed it.

Note that the first instance of Line 12 comes from the user manually copying it over from the reviewer.
The second instance of 12 comes from the label \emph{in your draft revision}.
This label should be set by you right after you read the review (most likely, this is a good use of your time, too).
Your rebuttal letter will automatically keep track of where the changes you make end up being in the revised manuscript as long as you don't delete the original command: \texttt{$\backslash$llabel\{ tag \}}.

If you need to put the line number directly in the text, you can simply use \texttt{$\backslash$ref\{ prefix tag \}}. 
Please note that there will be a prefix added to your tag from the other document, which you can change in \texttt{$\backslash$externaldocument[prefix]\{ path to document \}} in \texttt{preamble.tex}.

\lreviewer{Line 50}{}{This doesn't make any sense to me.}
\label{sec:test}
Notice that this uses the same macro, \texttt{$\backslash$lreviewer}, but leaves the second input blank.

\section{Reviewer Two}

\reviewer{Line 50}{This statement is confusing as written.}

Thank you. We have removed this accordingly. This agrees with Comment \ref{sec:test}.

This was written with the macro, \texttt{$\backslash$reviewer}, that only takes two inputs, i.e. it does not offer any line tracking.

\reviewer{Line 99}{Please check Schrodinger's cat.}

\begin{quote}
The cat in the box is \DIFdelbegin \DIFdel{dead}\DIFdelend \DIFaddbegin \DIFadd{alive}\DIFaddend .
\begin{align}
E &= mc^2 \\
m\cdot \DIFdelbegin \DIFdel{a=F}\DIFdelend \DIFaddbegin \DIFadd{v=p}\DIFaddend .
\end{align}
\end{quote}

You can use the \texttt{quote} and \texttt{latexdiff} packages to format changes to the text like this as well. 

\end{document}